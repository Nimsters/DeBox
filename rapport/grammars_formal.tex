\documentclass[BA.tex]{subfiles}
\begin{document}

\sss{Formal syntax}
\chk{As mentioned in the introduction}, one of the goals of this project
 was to define a formal syntax for proofs that was consistent with the
 \bp\ syntax\cite{boxhelp}. The simplest way to achieve consistency
 was to base \our\ formal syntax directly on the syntax of \bp.
 The resulting syntax for foumulas is given in \fig{fform}, where we see
 that implications (\verb+=>+) are right-associative, while conjunctions 
 (\verb+/\+) and disjunctions (\verb+\/+) are left-associative. It further
 shows the precedence levels, with negation (\verb+~+) binding the tightest
 and implication binding the lightest. The symbols for logical operators are
 represented by easily typed\footnote{By virtue of being available on 
 standard western computer keyboards.} characters, and match the
 operator specifications of \bp.

\begin{figure}[t]
\subfile{grammar_formal_formulae}
\caption{A formal syntax for formulae, based on \bp\ syntax.}
\label{fform}
\end{figure}

%\noindent
 An extention for this syntax is found in \fig{fproof}, which shows a formal
 syntax for proofs based directly on the \bp\ syntax.
 The essential components of a proof are the sequent to be
 proven and the steps of deduction by which it is done. 
 As seen in \fig{fproof}, the sequent in \our\ formal syntax
 follows standard mathematical notation, with the exception that the
 turnstyle, like the \chk{other} logical operators, is represented by
 easily typed characters (\verb+|-+) rather than the standard symbol 
 (\(\vd\)).
 
 The deduction steps themselves are written using 
 Fitch-style notation\cite{boxhelp},
 and are a list of lines and boxes. Such a list always ends with a 
 conclusion consisting of a formula
 and a deductive argument; each line consist of a conclusion followed 
 by a line-reference; each box is a list of steps that starts with an 
 assumption and, as always, ends with a conclusion, all of wich is in
 parenthesis followed by a box reference; an argument is either stating
 the formula a premise or stating by which rule and -- if any -- required 
 references the formula is deduced; the rules are grouped according to the
 pattern of their required references. 
 
 The rest of the syntax further 
 specifies the details of the \bp\ syntax, e.g. spacing requirements, and
 the concrete formulation of the start of the proof, which includes starting
 each proof with the string \f{\%abbrev} and a titlei, and listing all atoms
 in both curly and square
 brackets before and after stating the sequent, respectively.


\begin{figure}[H]
\subfile{grammar_formal_proofs}
\caption{A formal syntax for proofs, based on \bp\ syntax.}
\label{fproof}
\end{figure}

\sss{Design decisions: Operator associativity and proof composition}

The syntax shown in \fig{fform} complies with the associativity 
conventions stated in \emph{Logic in Computer Science}\cite[p.~5]{hr}, which
only specifies that implications are right-associative, but there is
general disagreement among textbooks on the associativity of conjunctions
and disjunctions. Some textbooks specify 
\emph{left}-associativity\cite[p.~46]{disc}, while others specify that 
conjunctions and disjunctions are \emph{right}-associative like 
implications\cite[p.~11]{math}\cite[p.~5]{calc}. The difference between
left- and right-associative conjunctions and disjunctions is the
interpretation of forulae like \(p \land q \land r\): when conjunctions are
\emph{left}-associative it is interpreted as \((p \land q) \land r\), but
when \emph{right}-associative, the interpretation is \(p \land (q \land r)\).
Since \(\land\) and \(\lor\) are associative operators, the two 
interpretations are logically equivalent. 
As a result, neither the validity nor the truth value of
\(p \land q \land r\) differs depending on the side to which they
associate, whereas \(p \ra (q \ra r)\) is \emph{not} logically equivalent
to \((p \ra q) \ra r\). Perhaps that is why textbooks agree on implications
being right-associative while the disagreement concerning conjunctions and
disjunctions seems to go unresolved.

Textbooks also disagree on wether conjunctions and disjunctions
are at the same precedence level\cite{hr} or conjunction bind tighter than
disjunctions\cite{math}\cite{calc}\cite{disc}. They do, however, all
agree that associativity and precedence conventions increase readability
by reducing the need for parenthesis while ensuring unambiguous
interpretation of 
formulae\cite[p.~5]{hr}\cite[p.~46]{disc}\cite[p.~10]{math}\cite[p.~5]{calc}.
Both associativity and precedence levels in the formal syntax was chosen 
to match those of \bp\ for consistency and because the \bp\ syntax separates
conjunctions and disjunctions with regard to precedence, thus ensuring the
least possible need for parenthesis in formulae.

 Basing the formal syntax for formulae on the \bp\ syntax meant few
 deviations from textbook standards, but the \bp\ syntax for proofs lies 
 somewhat further removed from standard notation due to its implementation.
 The additional requirements feature most strongly in the beginning of 
 proofs, as described above, while the base structure closely follows 
 standard Fitch-style notation. This last point was one of two main
 reasons behind the decision to fully adopt the \bp\ syntax as our formal
 syntax for proofs; the other was the expectation that students will do
 the majority of ther formal notation proofs in \bp\ anyway, since it has
 a pretty-printer which renders proofs very nicely in the style found in
 \emph{Logic in Computer Science}\chk{\cite{hr}}. Adding yet another formal
 syntax to the two already on the syllabus -- textbook and \bp\ notation --
 seemed unnecessary, and might risk confusing students rather than further
 their understanding. Instead, \we\ fully adopted the \bp\ syntax, seeing
 the base structure as \chk{a fine} foundation for the abstract syntax.

\sss{Abstract syntax}

The abstract syntax is implemented as types and datatypes in \s{Standard ML},
 and the full code can be found in \app{syntax}. 

At top-level, the abstract syntax follows the base structure of the
 \bp\ surface syntax, with a proof being represented by a title, a 
 sequent and a list of proof steps. This is implemented as the type
 \f{proof}, shown in \lst{../Proof.sml2828}, with a \f{string} 
 \chk{representing} the title.
 
\codepart{../Proof.sml}{Implementation of the abstract syntax for\
proofs \chk{at top level}}{28}{28}
 
 The abstract syntax for proof steps, however, does not match the surface
 syntax directly. Instead, there are only single-line steps, implemented
 as the type \f{proofstep} composed of a 
 \f{formula option}, a \f{rule}, a (possibly empty) \f{reference list}, 
 and a \f{string}. Thus, assumptions and discharges of assumptions are
 considered independent steps rather than composit parts of a box-step, as
 are all steps that occur between two such steps.
 The implementation is shown in 
 \lst{../Proof.sml2424} and ensures that all proofs are represented
 without any boxes \chk{or their nesting to arbitrary depth}.

\codepart{../Proof.sml}{Implementation of the abstract syntax for proof\
steps}{24}{24}

 This flattened structure, however, requires that the stating
 of premises, assumptions, and discharges are represented by rules of their
 own, rather than separate step constructors. Thus, the rule set is 
 implemented as the datatype \f{rule} with a nullary constructor for each
 of the inference rules known from propositional logic as well as for
 stating premises (\f{Prm}), assumptions (\f{Ass}), and discharges of 
 assumptions (\f{Dis}). 
 The definition of the \f{rule} datatype is shown in \lst{../Proof.sml1020}.

\codepart{../Proof.sml}{Implementation of the abstract syntax for rules}{10}
{20}

While boxes are not represented in the abstract syntax, the distinction
 between line- and box-\emph{references} is kept.
 These two \chk{kinds} of references are implemented as constructors of the 
 \f{reference} datatype; the \f{Line} 
 constructor takes a \f{string} as its argument wheras the \f{Box}
 constructor takes a \f{(string * string)} tuple, as shown in 
 \lst{../Proof.sml2222}.

\codepart{../Proof.sml}{Implementation of the abstract syntax for\
references}{22}{22}

Like the top-level \f{proof} type, the implementation of the abstract syntax
 for sequents and formulae follows the \bp\ surface syntax closely.
 Formulae are implemented as the recursive datatype \f{formula}, shown in
 \lst{../Proof.sml38}, while sequents are implemented as the \f{sequent} 
 type consisting of a (possibly empty) \f{formula list} and a \f{formula},
 shown in \lst{../Proof.sml2626}.

\codepart{../Proof.sml}{Implementation of the\
abstract syntax for formulae}{3}{8}

 As \lst{../Proof.sml38} shows, the \f{formula} datatype has a
 constructor for each \chk{logical}
 operator as well as for atoms and absurdity
 (\f{BOT}); the \f{BOT} constructor takes no arguments, the \f{Atom}
 constructor takes a \f{string}, the \f{NEG} constructor takes a \f{formula}
 as its argument,
 and the rest of the constructors all take a \f{(formula * formula)} tuple.

\codepart{../Proof.sml}{Implementation of the\
abstract syntax for formulae}{26}{26}

\sss{Design decisions: Feedback as a teaching tool}
The purpose of the tool \we\ have develped is to teach students to write 
 valid proofs \chk{in} propositional logic. 
 \chk{Giving concise and relevant feedback
 on errors is a big part of teaching any subject, and no less so when it
 comes to Logic in Computer Science.} Since \chk{\bp 's main function is to
 \emph{verify} proofs}, its syntax is designed to represent \emph{correct}
 proofs. As a result, the syntax is very strict, and making mistakes results
 in Twelf-generated error messages, which ``may sometimes look 
 cryptic''\cite[`Dealing with errors']{boxhelp}. Instead of halting on all
 errors during validation, \our\ tool handles many of them while completing 
 the validation process and giving continuous feedback. In order to 
 provide this feedback, the abstract syntax needs to represent erroneous 
 proofs as well as valid ones; hence the decision to implement 
 \f{proofstep} as one specified type, which fits all combinations of rules
 and reference patterns. With this design, the abstract syntax accepts both
 correctly composed proof steps and steps with too many or too few 
 references, references of the wrong kind, no inferred formula, and an
 inferred formula when there should have been none. The last case relates
 to assumption discharges, which are considered steps in their own right,
 and no longer dependent on a box structure; since boxes may be written
 incorrectly by students -- e.g. not closed, or two closed at the same time
 -- proper feedback depends on assumptions and discharges appearing as
 steps even when a properly constructed box does not. The chosen
 solution was, as mentioned above, to flatten the structure and replace
 boxes by their component steps.

\end{document}
