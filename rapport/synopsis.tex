\documentclass[a4paper]{article}

\usepackage{titlesec, fancyhdr, graphicx, float, nameref, framed, listings}
\usepackage{amsmath, enumerate, marvosym, fullpage, verbatim, subfiles}
\usepackage[hidelinks]{hyperref}
\usepackage[T1]{fontenc}
\usepackage[utf8]{inputenc}
\usepackage[usenames, dvipsnames, svgnames, table]{xcolor}
\usepackage[nottoc]{tocbibind}

\author{Io Boye Pinnerup}
\date{}

\newcommand{\f}[1]{\texttt{#1}}
\newcommand{\s}[1]{\textsc{#1}}
\newcommand{\ra}{\rightarrow}
\newcommand{\Ra}{\Rightarrow}
\newcommand{\LR}{\Leftrightarrow}
\newcommand{\txt}[1]{\textnormal{#1}}
\renewcommand{\.}{\cdot}
\renewcommand{\ss}[1]{\subsection{#1}}
\newcommand{\sss}[1]{\subsubsection{#1}}
\newcommand{\question}[1]{{\small \textcolor{Grey}{#1}}}

\pagestyle{fancy}
\fancyhead{}
\fancyfoot{}
\fancyhead[l]{I. B. Pinnerup}
\fancyhead[r]{\today}
\fancyfoot[c]{Page \thepage\ of \pageref{LastBody}}
\setlength{\headheight}{15pt}
\setlength{\headsep}{15pt}

\lstset{
    basicstyle=\footnotesize\ttfamily,
    keywordstyle = \color{Green},
    commentstyle = \color{Grey},
    identifierstyle = \color{Blue},
    stringstyle = \color{Purple},
    numbers = left,
    numberstyle = \tiny,
    stepnumber=1,
    showstringspaces = false,
    frame = single,
    captionpos = b,
    rulecolor = \color{Black},
    breaklines=true,
}

\newcommand{\code}[2]{\lstinputlisting[caption=\texttt{#2}, label=#1]{#1}}
\newcommand{\codepart}[4]{\lstinputlisting[caption=#2, firstline=#3,
                          lastline=#4, firstnumber=#3, label=#1#3#4]{#1}}
\newcommand{\coutput}[2]{\lstinputlisting[caption=\texttt{#2}, label=#1,
                        numbers=none, deletekeywords={*},
                        identifierstyle=\color{Black},
                        keywordstyle=\color{Black}]{#1}}
\newcommand{\fig}[3][]{
    \begin{figure}[H]
    \centering
    \includegraphics[width=\textwidth]{#2}
    \caption{#1}
    \label{#3}
    \end{figure}
}


\title{Natural deduction proofs in (pseudo) natural language}
\fancyhead[c]{}
\titlelabel{}
\begin{document}
\maketitle
\section{Problem statement}
    Is it possible to develop a tool for writing natural deduction proofs
    in a subset of the English language that will read as natural language
    (from here on, this subset will just be named `natural language'),
    and have these proofs validated? Additionally, is it feasible to have
    a round-tripping ability such that {\bf formal $\ra$ natural language
    $\ra$ formal} is the identity function at the level of abstract syntax,
    and {\bf natural language $\ra$ formal $\ra$ natural language} is the
    identity function modulo lexical aspects such as whitespace and
    punctuation.

\section{Scope}
    This project will limit itself to natural deduction proofs of
    \emph{propositional} logic, and the validation is expected to be done
    using the existing tool \s{BoxProver}\cite{box}. Furthermore, the tool
    to be developed is specifically intended for use in the course `Logic
    in computer science' at the University of Copenhagen, and for  
    pedagogical reasons, the atoms used in the natural language syntax
    will be characters rather than specified declarative sentences.

\section{Motivation}
    The course `Logic in computer science' rather closely follows the 
    structure outlined in the textbook `Logic in computer science' by
    Huth \& Ryan\cite{hr}. The introduction to formal logic is based on
    reasoning about construction of (valid) arguments in natural 
    language\cite[pp.~1-2]{hr}, which forms the basis for the presentation
    of a formal notation using the symbols 
    \(\neg, \lor, \land, \txt{ and } \ra\)\cite[p.~4]{hr}. 
    The \emph{meaning} of these symbols is presented in natural language,
    but with the exception of a few simple examples, the rest of the section 
    on propositional logic, including the rules for natural deduction,
    is presented solely in formal (i.e. symbol) notation.

    Presenting students with a tool for writing natural deduction proofs
    in natural language is hoped to bridge the potential gap between
    understanding the structure and rules of natural deduction and
    understanding the \emph{meaning} of the steps in such a process as well
    as the conclusion reached. Furthermore, it is hoped the the
    presentation of such proofs in natural language will emphasize the
    \emph{independence} of the reached conclusion from the semantic 
    content of atomic declarative sentences; \(p, p \ra q \vdash q\)
    regardless
    of what declarative sentences one may choose $p$ and $q$ to represent.

\section{Tasks}

\begin{itemize}
\item 
    Formulation of a concrete {\bf natural language} syntax for formulae and
    proofs by natural deduction.
    \begin{description}
        \item[Criteria:]~ \\ \vspace{-5 mm}
            \begin{itemize}
                \item Should read as natural language: Ideally compliant
                with Attempto Controlled English, otherwise arguably
                consistent with standard English.
                \item Should be arguably consistent with commonly used 
                mathematical terminology.
            \end{itemize}
        \item[Time needed:] 2-3 weeks.
    \end{description}

\item
    Formulation of a concrete {\bf formal} syntax and corresponding abstract
    syntax for propositional formulae and proofs.
    \begin{description}
        \item[Criteria:]~ \\ \vspace{-5 mm}
            \begin{itemize}
                \item Should be consistent with \s{BoxProver}\cite{box}
                syntax.
            \end{itemize}
        \item[Time needed:] 1-2 weeks.
    \end{description}

\item
    Design, implementation and test of formal $\longleftrightarrow$ natural
    language conversions.
    \begin{description}
        \item[Criteria:]~ \\ \vspace{-5 mm}
            \begin{itemize}
                \item Should convert correctly from {\bf formal} to
                {\bf natural language}.
                \item Should convert correctly form {\bf natural language}
                to {\bf formal}.
                \item Should, if possible, display round-trip ability as
                described in the problem statement.
            \end{itemize}
        \item[Time needed:] 4-5 weeks.
    \end{description}

\item
    Development of a library of formulae and proofs for test and validation,
    including method and tool support for managing and adding to such a 
    library.
    \begin{description}
        \item[Criteria:]~ \\ \vspace{-5 mm}
            \begin{itemize}
                \item Tool support for regression testing.
                \item Should test a collection of examples from relevant
                textbooks, specificaly Huth \& Ryan\cite{hr}.
            \end{itemize}
        \item[Time needed:] 1 week.
    \end{description}
\end{itemize}

\section{Product}
The project will result in a report documenting the syntax of the 
natural language, the formal and abstract syntax of propositional 
logic, and the design, implementation, and testing of a conversion tool 
between the two.

\label{LastBody}
\bibliographystyle{plain}
\bibliography{Bibliography}
\end{document}
