\documentclass[a4paper]{article}

\usepackage{titlesec, fancyhdr, graphicx, float, nameref, framed, listings}
\usepackage{amsmath, enumerate}
\usepackage[hidelinks]{hyperref}
\usepackage[T1]{fontenc}
\usepackage[utf8]{inputenc}
\usepackage[usenames, dvipsnames, svgnames, table]{xcolor}
\usepackage[nottoc]{tocbibind}
\usepackage{fullpage, verbatim}

\author{Io Boye Pinnerup}
\date{}

\newcommand{\f}[1]{\texttt{#1}}
\newcommand{\s}[1]{\textsc{#1}}
\newcommand{\ra}{\rightarrow}
\newcommand{\Ra}{\Rightarrow}
\newcommand{\LR}{\Leftrightarrow}
\newcommand{\txt}[1]{\textnormal{#1}}
\renewcommand{\.}{\cdot}
\renewcommand{\ss}[1]{\subsection{#1}}
\newcommand{\sss}[1]{\subsubsection{#1}}
\newcommand{\question}[1]{{\small \textcolor{Grey}{#1}}}

\pagestyle{fancy}
\fancyhead{}
\fancyfoot{}
\fancyhead[l]{I. B. Pinnerup}
\fancyhead[r]{\today}
\fancyfoot[c]{Page \thepage\ of \pageref{LastBody}}
\setlength{\headheight}{15pt}
\setlength{\headsep}{15pt}

\lstset{
    basicstyle=\footnotesize\ttfamily,
    keywordstyle = \color{Green},
    commentstyle = \color{Grey},
    identifierstyle = \color{Blue},
    stringstyle = \color{Purple},
    numbers = left,
    numberstyle = \tiny,
    stepnumber=1,
    showstringspaces = false,
    frame = single,
    captionpos = b,
    rulecolor = \color{Black},
    breaklines=true,
}

\newcommand{\code}[2]{\lstinputlisting[caption=\texttt{#2}, label=#1]{#1}}
\newcommand{\codepart}[4]{\lstinputlisting[caption=#2, firstline=#3,
                          lastline=#4, firstnumber=#3, label=#1#3#4]{#1}}
\newcommand{\coutput}[2]{\lstinputlisting[caption=\texttt{#2}, label=#1,
                        numbers=none, deletekeywords={*},
                        identifierstyle=\color{Black},
                        keywordstyle=\color{Black}]{#1}}
\newcommand{\fig}[3][]{
    \begin{figure}[H]
    \centering
    \includegraphics[width=\textwidth]{#2}
    \caption{#1}
    \label{#3}
    \end{figure}
}

\titlelabel{}
\newcommand{\tit}{Natural deduction proofs in (pseudo) natural language}
\title{\tit}
\fancyhead[c]{\tit}
\newcommand{\we}{we}
\newcommand{\We}{We}
\newcommand{\our}{our}
\newcommand{\us}{us}
\newcommand{\bp}{\s{BoxProver}}
\newcommand{\FIX}[1]{\textcolor{Red}{FIX THIS: #1}}
%\newcommand{\FIX}[1]{#1}
\newcommand{\chk}[1]{\textcolor{Blue}{Check this: #1}}
%\newcommand{\chk}[1]{#1}
\lstset{language=ML, 
emph={BasicIO, TextIO, List, Int, Lexing, Lexer, Parser, Validation,
Nonstdio, Auxiliaries},
emphstyle={\color{Olive}}}

\begin{document}
\pagenumbering{Alph}
\begin{titlepage}
\maketitle
\thispagestyle{empty}
\end{titlepage}

\tableofcontents
\thispagestyle{empty}
\pagebreak
\lstlistoflistings
%\listoffigures
%\pagebreak
\pagenumbering{arabic}

\ss{Resumé}

\ss{Abstract}

\section{Introduction} %Remove splits into subsections
\subsection{Problem statement}
    Is it possible to develop a tool for writing natural deduction proofs
    in a subset of the English language that will read as natural language
    (from here on, this subset will just be named `natural language'),
    and have these proofs validated? Additionally, is it feasible to have
    a round-tripping ability such that {\bf formal $\ra$ natural language
    $\ra$ formal} is the identity function at the level of abstract syntax,
    and {\bf natural language $\ra$ formal $\ra$ natural language} is the
    identity function modulo lexical aspects such as whitespace and
    punctuation.

\subsection{Scope}
    This project will limit itself to natural deduction proofs of
    \emph{propositional} logic, and the validation is expected to be done
    using the existing tool \s{BoxProver}\cite{box}. Furthermore, the tool
    to be developed is specifically intended for use in the course `Logic
    in computer science' at the University of Copenhagen, and for  
    pedagogical reasons, the atoms used in the natural language syntax
    will be characters rather than specified declarative sentences.

\subsection{Motivation}
    The course `Logic in computer science' rather closely follows the 
    structure outlined in the textbook `Logic in computer science' by
    Huth \& Ryan\cite{hr}. The introduction to formal logic is based on
    reasoning about construction of (valid) arguments in natural 
    language\cite[pp.~1-2]{hr}, which forms the basis for the presentation
    of a formal notation using the symbols 
    \(\neg, \lor, \land, \txt{ and } \ra\)\cite[p.~4]{hr}. 
    The \emph{meaning} of these symbols is presented in natural language,
    but with the exception of a few simple examples, the rest of the section 
    on propositional logic, including the rules for natural deduction,
    is presented solely in formal (i.e. symbol) notation.

    Presenting students with a tool for writing natural deduction proofs
    in natural language is hoped to bridge the potential gap between
    understanding the structure and rules of natural deduction and
    understanding the \emph{meaning} of the steps in such a process as well
    as the conclusion reached. Furthermore, it is hoped the the
    presentation of such proofs in natural language will emphasize the
    \emph{independence} of the reached conclusion from the semantic 
    content of atomic declarative sentences; \(p, p \ra q \vdash q\)
    regardless
    of what declarative sentences one may choose $p$ and $q$ to represent.

\subsection{Results}

\subsection{Example}

\section{Tools and programming language} % Necessary? Other title?

\newpage

\section{Grammars}
\FIX{SOMETHING BRIEF ABOUT CONTEXTFREE GRAMMARS}
\ss{Propositional logic as a formal language}
\subfile{grammars_formal}

\ss{Grammar for proofs in natural language}
\subfile{grammars_proofs}

\subfile{grammars_formulae}

\section{Parsing and unparsing}
\ss{Lexer}
\ss{Parser}
\ss{Unparsing to natural language}
\ss{Unparsing to \s{BoxProver} syntax}
\ss{Decisions on specific implementation details}
\sss{Handling boxes}

\section{Validation}
\ss{The validation module}
\ss{Decisions on specific implementation details}

\section{Evaluation}
\ss{Qualitative evaluations}
\sss{Consistency with standard English}
\sss{Consistency with common mathematical terminology}
\sss{Consistency with \s{BoxProver} syntax}
\ss{Testing of code}
\sss{Unittesting}
\sss{Usecases}
\sss{Roundtripping}

\section{Discussion}

\section{Conclusions}


\label{LastBody}
\pagebreak
\pagenumbering{roman}
\fancyfoot[c]{Page \thepage\ of \pageref{LastPage}}
%\nocite{*}
\bibliographystyle{plain}
\bibliography{Bibliography}
\label{LastBio}
\pagebreak
\appendix
\titlelabel{\thetitle . \quad}
\section*{Appendices}\label{FirstApp}
\addcontentsline{toc}{section}{Appendices}
\renewcommand \thesubsection{\alph{subsection}}
\section{Code}
\ss{\tt{Proof.sml}}\label{Proof}
\codeapp{../Proof.sml}

\ss{\tt{Lexer.lex}}\label{Lexer}
\codeapp{../Lexer.lex}

\ss{\tt{Parser.grm}}\label{Parser}
\codeapp{../Parser.grm}

\ss{\tt{Main.sml}}\label{Main}
\codeapp{../Main.sml}

\ss{\tt{Validation.sml}}\label{Validation}
\codeapp{../Validation.sml}

\ss{\tt{Validation.sig}}\label{Validation}
\codeapp{../Validation.sig}

\ss{\tt{Auxiliaries.sml}}\label{Auxiliaries}
\codeapp{../Auxiliaries.sml}

\label{LastApp}
\label{LastPage}
\end{document}
