\documentclass[a4paper]{article}

\usepackage{titlesec, fancyhdr, graphicx, float, nameref, framed, listings}
\usepackage{amsmath, enumerate, marvosym, fullpage, verbatim, subfiles}
\usepackage[hidelinks]{hyperref}
\usepackage[T1]{fontenc}
\usepackage[utf8]{inputenc}
\usepackage[usenames, dvipsnames, svgnames, table]{xcolor}
\usepackage[nottoc]{tocbibind}

\author{Io Boye Pinnerup}
\date{}

\newcommand{\f}[1]{\texttt{#1}}
\newcommand{\s}[1]{\textsc{#1}}
\newcommand{\ra}{\rightarrow}
\newcommand{\Ra}{\Rightarrow}
\newcommand{\LR}{\Leftrightarrow}
\newcommand{\txt}[1]{\textnormal{#1}}
\renewcommand{\.}{\cdot}
\renewcommand{\ss}[1]{\subsection{#1}}
\newcommand{\sss}[1]{\subsubsection{#1}}
\newcommand{\question}[1]{{\small \textcolor{Grey}{#1}}}

\pagestyle{fancy}
\fancyhead{}
\fancyfoot{}
\fancyhead[l]{I. B. Pinnerup}
\fancyhead[r]{\today}
\fancyfoot[c]{Page \thepage\ of \pageref{LastBody}}
\setlength{\headheight}{15pt}
\setlength{\headsep}{15pt}

\lstset{
    basicstyle=\footnotesize\ttfamily,
    keywordstyle = \color{Green},
    commentstyle = \color{Grey},
    identifierstyle = \color{Blue},
    stringstyle = \color{Purple},
    numbers = left,
    numberstyle = \tiny,
    stepnumber=1,
    showstringspaces = false,
    frame = single,
    captionpos = b,
    rulecolor = \color{Black},
    breaklines=true,
}

\newcommand{\code}[2]{\lstinputlisting[caption=\texttt{#2}, label=#1]{#1}}
\newcommand{\codepart}[4]{\lstinputlisting[caption=#2, firstline=#3,
                          lastline=#4, firstnumber=#3, label=#1#3#4]{#1}}
\newcommand{\coutput}[2]{\lstinputlisting[caption=\texttt{#2}, label=#1,
                        numbers=none, deletekeywords={*},
                        identifierstyle=\color{Black},
                        keywordstyle=\color{Black}]{#1}}
\newcommand{\fig}[3][]{
    \begin{figure}[H]
    \centering
    \includegraphics[width=\textwidth]{#2}
    \caption{#1}
    \label{#3}
    \end{figure}
}


\fancyhead[c]{Labnotes}
\titlelabel{}

\begin{document}

\ss{March 31st}
The grammar for NL formulas so far:
\begin{quote}
\ttfamily
PROPOSITION $\ra$ IMPLICATION | PROP\\
IMPLICATION $\ra$ \txt{if } PROP \txt{ is provable then so is } PROP\\
\(~~~~~~~~~~~~~~~\) |
\txt{if } PROP \txt{ is provable then so is the proposition that } IMP'\\
IMP' \(~~~~~~~\)$\ra$ IMPLICATION | IMP \\
PROP \(~~~~~~~\)$\ra$ SIMPLE | LIST \\
LIST \(~~~~~~~\)$\ra$ PAIR\txt{, } PAIR\txt{, and also } PAIR |
                      PAIR\txt{, } PAIR\txt{, or else } PAIR\\
\(~~~~~~~~~~~~~~~\) | PAIR\txt{, } LIST\\
SIMPLE \(~~~~~\)$\ra$ PAIR \txt{ and also } PAIR |
                      PAIR \txt{ or else } PAIR  | PAIR\\
PAIR \(~~~~~~~\)$\ra$ SINGLE \txt{ and }  SINGLE | SINGLE \txt{ or } SINGLE
                    | SINGLE | IMP\\
IMP \(~~~~~~~~\)$\ra$ SINGLE \txt{ implies } SINGLE
                    | SINGLE \txt{ does not imply } SINGLE\\
SINGLE \(~~~~~\)$\ra$ \txt{the statement ``}PROPOSITION\txt{''} |
                      \txt{the negation of ``}PROPOSITION\txt{''}\\
\(~~~~~~~~~~~~~~~\) | NEGATION | {\bf atom}\\
NEGATION \(~~~\)$\ra$ \txt{not-}NEGATION | \txt{not-}{\bf atom}\\
\end{quote}

\sss{Quotation marks}
If the grammar is to account for correctly alternating quotation marks,
then everything will have to be repeated and the two grammar "sets" should
only cross in the \f{formula} and \f{negation} constructions:
\begin{quote}
\ttfamily
PROPOSITION $\ra$ IMPLICATION | PROP\\
$\vdots$\\
SINGLE \(~~~~~\)$\ra$ \txt{the statement ``}*PROPOSITION\txt{''} |
                      \txt{the negation of ``}*PROPOSITION\txt{''}\\
\(~~~~~~~~~~~~~~~\) | NEGATION | {\bf atom}\\
NEGATION \(~~~\)$\ra$ \txt{not-}NEGATION | \txt{not-}{\bf atom}\\
~\\
*PROPOSITION $\ra$ IMPLICATION | PROP\\
$\vdots$\\
*SINGLE \(~~~~~\)$\ra$ \txt{the statement `}PROPOSITION\txt{'} |
                      \txt{the negation of `}PROPOSITION\txt{'}\\
\(~~~~~~~~~~~~~~~\) | NEGATION | {\bf atom}\\
\end{quote}

\sss{Linguistic negation}
The linguistic negations of pairs:
\begin{quote}
not both p and q are valid\\
neither p nor q is valid\\
\end{quote}
does not read properly when used within a longer formula:
\begin{quote}
    not both p and q are valid and also r\\
    p and also neither q nor r is valid\\
\end{quote}
Instead, these formulas should read:
\begin{quote}
    not both p and q are valid but r is\\
    p is valid but neither q nor r is
\end{quote}
To account for this, one would have to make the grammar a lot more complex
with special cases for \f{and also}, \f{or else} and \f{implication} for 
every possible combination of negated and non-negated sub-propositions.

Linguistic negation of lists and \f{and also}/\f{or else} constructions
seem difficult to do without introducing ambiguity.
Must research further on linguistic negation to determine how exactly to 
handle these cases.

\ss{April 8th}
Some changes still need to be done on the grammar.
\begin{itemize}
    \item For \s{implication}: to ensure the distinction between the two
    types, a \s{prop} should no longer convert to \s{imp}.

    \item For \s{simple}:
        \begin{itemize}
            \item Not both first and second child should be \s{single}
            \item For first child: \s{imp} should not be a possibily unless
            in brackets (when read, it itroduces ambiguity despite the
            distance created by `and also' and `or else').

            \item For second child: unbracketed \s{implication} should be a
            possibility.
        \end{itemize}

    \item For bracketing: only complex statements should be bracketed.

    \item For negation statements: atom- and not-negations should never be
    negated using statements

\end{itemize}

The new grammar looks as follows:

\begin{quote}
\ttfamily
PROPOSITION $\ra$ COMPLEX | PAIR | NEGATION | UNIT\\
PROP \(~~~~~~~\)$\ra$ LIST | MULTI | PAIR\\
PROP' \(~~~~~~\)$\ra$ LIST | MULTI | PAIR | SINGLE\\
PROP* \(~~~~~~\)$\ra$ COMPLEX | PAIR | NEGATION\\
COMPLEX \(~~~~\)$\ra$ IMPLICATION | LIST | MULTI\\
IMPLICATION $\ra$ SINGLE \txt{ implies } SINGLE
                    | SINGLE \txt{ implies that } IMPLICATION\\
\(~~~~~~~~~~~~~~~\) |
\txt{if } PROP\txt{ is provable then so is } PROP'\\
\(~~~~~~~~~~~~~~~\) |
\txt{if } PROP' \txt{ is provable then so is } PROP\\
\(~~~~~~~~~~~~~~~\) |
\txt{if } PROP \txt{ is provable then so is the proposition that } 
IMPLICATION\\
LIST \(~~~~~~~\)$\ra$ ELEMENT\txt{, } ELEMENT\txt{, and also } ELEMENT |
                      ELEMENT\txt{, } ELEMENT\txt{, or else } ELEMENT\\
\(~~~~~~~~~~~~~~~\) | ELEMENT\txt{, } LIST\\
MULTI \(~~~~~~\)$\ra$ PAIR \:\txt{ and also } ELEMENT |
                      SINGLE \txt{ and also } ELEMENT' |\\
\(~~~~~~~~~~~~~~~\) | PAIR \txt{ or else } ELEMENT  |
                      SINGLE \txt{ or else } ELEMENT'\\
ELEMENT \(~~~~\)$\ra$ PAIR | IMPLICATION | SINGLE\\
ELEMENT' \(~~~\)$\ra$ PAIR | IMPLICATION\\
PAIR \(~~~~~~~\)$\ra$ SINGLE \txt{ and }  SINGLE | SINGLE \txt{ or } SINGLE\\
SINGLE \(~~~~~\)$\ra$ STAT | UNIT | NEGATION\\
STAT \(~~~~~~~\)$\ra$ \txt{the statement ``}COMPLEX\txt{''}\\
NEGATION \(~~~\)$\ra$ \txt{the negation of ``}PROP*\txt{''}\\
NEG \(~~~~~~~~\)$\ra$ \txt{not-}UNIT\\
UNIT \(~~~~~~~\)$\ra$ NEG | {\bf atom}\\
\end{quote}
Question: Should \s{implication} remain as possible \s{element}? Does the 
Oxford comma remove the ambiguity?

\ss{April 10th}
Discovered several errors while implementing NL grammar:
\begin{itemize}
    \item \f{NEG} is superflous: \f{UNIT} = not-\f{UNIT} | {\bf atom}
    \item The grammar accepts both:
        \begin{quote}\f{SINGLE} and \f{IMPLICATION}\\
                     \f{SINGLE} and the statement \f{IMPLICATION}
        \end{quote}
        The same goes for or.
    \item Any list must either always start with a pair of the same type or
    never start with a pair of the same type. This should be reflected in 
    the grammar.
\end{itemize}
Further question: should \f{PAIR} be replaced Lists with only two elements
and thus be written "p, and also q or r"/ "p or q, and also r"?

\ss{April 11th}
Discovered another error: "if \f{PROP'} is provable then so is \f{PROP}"
should be "if \f{SINGLE} is provable then so is \f{PROP}. This means that
\f{PROP'} can be restructured to \f{PROP} | \f{SINGLE}.

\sss{Solution to structure of lists?}
If we extend the type of elements with:
\begin{quote}
    \f{EXT } $\ra$ \f{ELEMENT} | \f{EXT}, \f{ELEMENT}
\end{quote}
Then \f{LIST} could be reduced to:
\begin{quote}
    \f{LIST} $\ra$ \f{EXT}, and also \f{ELEMENT} 
                 | \f{EXT}, or else  \f{ELEMENT}
\end{quote}
(presuming we also replace \f{MULTI} with two-element lists. A similar
structure for triples could easily be constructed, but requires \f{EXT} to
be split up further).

This would preserve the structure of the nested \f{AND}s, and I think it
would be a lot easier to work with when parsing NL to symbolic.

\ss{April 12th}
Reworked the grammar:
\begin{quote}
\ttfamily
PROPOSITION $\ra$ COMPLEX | IMPLICATION | NEGATION | UNIT\\
PROP* \(~~~~~~\)$\ra$ COMPLEX | IMPLICATION | NEGATION\\
PROP' \(~~~~~~\)$\ra$ COMPLEX | SINGLE\\
PROP \(~~~~~~~\)$\ra$ COMPLEX | IMP\\
COMPLEX \(~~~~\)$\ra$ COMP | PAIR\\
COMP \(~~~~~~~\)$\ra$ LIST | MULTI\\
IMP \(~~~~~~~~\)$\ra$ \txt{the implication ``}IMPLICATION\txt{''}\\
IMPLICATION $\ra$ SINGLE \txt{ implies } SINGLE
                    | SINGLE \txt{ implies that } IMPLICATION\\
\(~~~~~~~~~~~~~~~\) |
\txt{if } SINGLE \txt{ is provable then so is } COMPLEX\\
\(~~~~~~~~~~~~~~~\) |
\txt{if } PROP \txt{ is provable then so is } PROP'\\
\(~~~~~~~~~~~~~~~\) |
\txt{if } PROP \txt{ is provable then so is the proposition that } 
IMPLICATION\\
LIST \(~~~~~~~\)$\ra$ EXTEND\txt{, and also } ELEMENT |
                      EXTEND\txt{, or else } ELEMENT\\
MULTI \(~~~~~~\)$\ra$ PAIR \:\txt{ and also } ELEMENT |
                      SINGLE \txt{ and also } ELEMENT'\\
\(~~~~~~~~~~~~~~~\) | PAIR \txt{ or else } ELEMENT  |
                      SINGLE \txt{ or else } ELEMENT'\\
EXTEND \(~~~~~\)$\ra$ EXT\txt{,} ELEMENT\\
EXT \(~~~~~~~~\)$\ra$ EXTEND | ELEMENT\\
ELEMENT \(~~~~\)$\ra$ PAIR | IMPLICATION | SINGLE\\
ELEMENT' \(~~~\)$\ra$ PAIR | IMPLICATION\\
PAIR \(~~~~~~~\)$\ra$ SINGLE \txt{ and }  SINGLE | SINGLE \txt{ or } SINGLE\\
SINGLE \(~~~~~\)$\ra$ STAT | NEGATION | UNIT\\
STAT \(~~~~~~~\)$\ra$ \txt{the statement ``}COMP\txt{''}\\
NEGATION \(~~~\)$\ra$ \txt{the negation of ``}PROP*\txt{''}\\
UNIT \(~~~~~~~\)$\ra$ \txt{not-}UNIT | {\bf atom}\\
\end{quote}

\ss{April 13th}
Noticed that \f{IMP} was missing as a possible first element in \f{MULTI}:
\begin{quote}
\ttfamily
MULTI \(~~~~~~\)$\ra$ PAIR \:\txt{ and also } ELEMENT |
                      SINGLE \txt{ and also } ELEMENT' |\\
\(~~~~~~~~~~~~~~~\) | PAIR \txt{ or else } ELEMENT  |
                      SINGLE \txt{ or else } ELEMENT'\\
\(~~~~~~~~~~~~~~~\) | IMP ~\txt{ or else } ELEMENT  |
                      IMP \txt{ and also } ELEMENT\\
\end{quote}

{\bf Questions:}
\begin{itemize}
    \item Should \f{IMP} be handled as \f{IMPLICATION} or as \s{STAT}? I.e.
    what makes more sense, that implications are always in a \f{MULTI} 
    regardless of what side of the operator they're on, or that leftside,
    and thus bracketed, implications are treated as statements, and thus 
    can be the first element in a \f{PAIR}?
    So far, I'm treating both implications equally.

    \item What to do with \f{MULTI} and \f{PAIR} with regards to \f{SINGLE}.
    Should \f{SINGLE} be exchanged for \f{UNIT}?
\end{itemize}

\ss{May ...}

\ss{May 13th}

\ss{May 15th}
{\bf Questions:}
  \begin{itemize}
    \item Concerning self-references:
      \begin{itemize}
        \item Should the grammar require that the string be marked as 
        separate from the rest of the sentence?
        \item Should there be other ways of marking than [...]
        \item Should there be checks and error reporting on marking?
      \end{itemize}
  \end{itemize}

{\bf Decisions:}
  \begin{itemize}
    \item Reference strings (including self-references) are represented by 
    the full string, including marking (square brackets only, so far).
  \end{itemize}

\label{LastBody}
\end{document}
