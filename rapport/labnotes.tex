\documentclass[a4paper]{article}

\usepackage{titlesec, fancyhdr, graphicx, float, nameref, framed, listings}
\usepackage{amsmath, enumerate, marvosym, fullpage, verbatim, subfiles}
\usepackage[hidelinks]{hyperref}
\usepackage[T1]{fontenc}
\usepackage[utf8]{inputenc}
\usepackage[usenames, dvipsnames, svgnames, table]{xcolor}
\usepackage[nottoc]{tocbibind}

\author{Io Boye Pinnerup}
\date{}

\newcommand{\f}[1]{\texttt{#1}}
\newcommand{\s}[1]{\textsc{#1}}
\newcommand{\ra}{\rightarrow}
\newcommand{\Ra}{\Rightarrow}
\newcommand{\LR}{\Leftrightarrow}
\newcommand{\txt}[1]{\textnormal{#1}}
\renewcommand{\.}{\cdot}
\renewcommand{\ss}[1]{\subsection{#1}}
\newcommand{\sss}[1]{\subsubsection{#1}}
\newcommand{\question}[1]{{\small \textcolor{Grey}{#1}}}

\pagestyle{fancy}
\fancyhead{}
\fancyfoot{}
\fancyhead[l]{I. B. Pinnerup}
\fancyhead[r]{\today}
\fancyfoot[c]{Page \thepage\ of \pageref{LastBody}}
\setlength{\headheight}{15pt}
\setlength{\headsep}{15pt}

\lstset{
    basicstyle=\footnotesize\ttfamily,
    keywordstyle = \color{Green},
    commentstyle = \color{Grey},
    identifierstyle = \color{Blue},
    stringstyle = \color{Purple},
    numbers = left,
    numberstyle = \tiny,
    stepnumber=1,
    showstringspaces = false,
    frame = single,
    captionpos = b,
    rulecolor = \color{Black},
    breaklines=true,
}

\newcommand{\code}[2]{\lstinputlisting[caption=\texttt{#2}, label=#1]{#1}}
\newcommand{\codepart}[4]{\lstinputlisting[caption=#2, firstline=#3,
                          lastline=#4, firstnumber=#3, label=#1#3#4]{#1}}
\newcommand{\coutput}[2]{\lstinputlisting[caption=\texttt{#2}, label=#1,
                        numbers=none, deletekeywords={*},
                        identifierstyle=\color{Black},
                        keywordstyle=\color{Black}]{#1}}
\newcommand{\fig}[3][]{
    \begin{figure}[H]
    \centering
    \includegraphics[width=\textwidth]{#2}
    \caption{#1}
    \label{#3}
    \end{figure}
}


\fancyhead[c]{Labnotes}
\titlelabel{}

\begin{document}

\ss{March 31st}
The grammar for NL formulas so far:
\begin{quote}
\ttfamily
PROPOSITION $\ra$ IMPLICATION | PROP\\
IMPLICATION $\ra$ \txt{if } PROP \txt{ is provable then so is } PROP\\
\(~~~~~~~~~~~~~~~\) |
\txt{if } PROP \txt{ is provable then so is the proposition that } IMP'\\
IMP' \(~~~~~~~\)$\ra$ IMPLICATION | IMP \\
PROP \(~~~~~~~\)$\ra$ SIMPLE | LIST \\
LIST \(~~~~~~~\)$\ra$ PAIR\txt{, } PAIR\txt{, and also } PAIR |
                      PAIR\txt{, } PAIR\txt{, or else } PAIR\\
\(~~~~~~~~~~~~~~~\) | PAIR\txt{, } LIST\\
SIMPLE \(~~~~~\)$\ra$ PAIR \txt{ and also } PAIR |
                      PAIR \txt{ or else } PAIR  | PAIR\\
PAIR \(~~~~~~~\)$\ra$ SINGLE \txt{ and }  SINGLE | SINGLE \txt{ or } SINGLE
                    | SINGLE | IMP\\
IMP \(~~~~~~~~\)$\ra$ SINGLE \txt{ implies } SINGLE
                    | SINGLE \txt{ does not imply } SINGLE\\
SINGLE \(~~~~~\)$\ra$ \txt{the statement ``}PROPOSITION\txt{''} |
                      \txt{the negation of ``}PROPOSITION\txt{''}\\
\(~~~~~~~~~~~~~~~\) | NEGATION | {\bf atom}\\
NEGATION \(~~~\)$\ra$ \txt{not-}NEGATION | \txt{not-}{\bf atom}\\
\end{quote}

\sss{Quotation marks}
If the grammar is to account for correctly alternating quotation marks,
then everything will have to be repeated and the two grammar "sets" should
only cross in the \f{formula} and \f{negation} constructions:
\begin{quote}
\ttfamily
PROPOSITION $\ra$ IMPLICATION | PROP\\
$\vdots$\\
SINGLE \(~~~~~\)$\ra$ \txt{the statement ``}*PROPOSITION\txt{''} |
                      \txt{the negation of ``}*PROPOSITION\txt{''}\\
\(~~~~~~~~~~~~~~~\) | NEGATION | {\bf atom}\\
NEGATION \(~~~\)$\ra$ \txt{not-}NEGATION | \txt{not-}{\bf atom}\\
~\\
*PROPOSITION $\ra$ IMPLICATION | PROP\\
$\vdots$\\
*SINGLE \(~~~~~\)$\ra$ \txt{the statement `}PROPOSITION\txt{'} |
                      \txt{the negation of `}PROPOSITION\txt{'}\\
\(~~~~~~~~~~~~~~~\) | NEGATION | {\bf atom}\\
\end{quote}

\sss{Linguistic negation}
The linguistic negations of pairs:
\begin{quote}
not both p and q are valid\\
neither p nor q is valid\\
\end{quote}
does not read properly when used within a longer formula:
\begin{quote}
    not both p and q are valid and also r\\
    p and also neither q nor r is valid\\
\end{quote}
Instead, these formulas should read:
\begin{quote}
    not both p and q are valid but r is\\
    p is valid but neither q nor r is
\end{quote}
To account for this, one would have to make the grammar a lot more complex
with special cases for \f{and also}, \f{or else} and \f{implication} for 
every possible combination of negated and non-negated sub-propositions.

Linguistic negation of lists and \f{and also}/\f{or else} constructions
seem difficult to do without introducing ambiguity.
Must research further on linguistic negation to determine how exactly to 
handle these cases.
\label{LastBody}
\end{document}
