\documentclass[BA.tex]{subfiles}
\begin{document}

\ss{Parser}
The grammar described in ``\nameref{nlgram}'' is implemented as a parser
 definition with embedded
 \sml\ actions, \f{Parser.grm}, which can be found in full in \app{Parser}
 and was used to generate the source code for
 a parser for natural deduction proofs
 written in natural language. It was generated with the tool \yac ,
 using the abstract syntax implemented in \f{\nameref{Proof}}
 (see \app{Proof}). Like the abstract syntax, the parser definition is
 written in \chk{\s{Standard ML}}\FIX{Add something here}.
 As seen in \lst{../Parser.grm125}, tokens are defined to
 represent all terminals of the grammar for proofs in natural language,
 and all non-terminals are named; precedence levels and
 associativity is specified for the logical operators; all tokens and
 non-terminals are assigned a type using the datatype and type definitions
 from \f{Proof.sml}, and a start symbol for the grammar is specified
 \chk{(l. 13)}.
 \codepart{../Parser.grm}{Definition of \f{tokens} and assignment
 of types.}{1}{25}
 Following this, the structure of the grammar is defined by specifying all
 productions in the grammar.

 %\codeapp{../Parser.grm}
 
\ss{Lexer}
The lexer was generated from the lexer definition found in \app{Lexer}.
\ss{Unparsing to \bp\ syntax}
\FIX{\ss{Unparsing to natural language}}
\FIX{\subfile{parsing_details}}
\end{document}
