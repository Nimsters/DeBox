\documentclass[a4paper]{article}

\usepackage{titlesec, fancyhdr, graphicx, float, nameref, framed, listings}
\usepackage{amsmath, enumerate, marvosym, fullpage, verbatim, subfiles}
\usepackage[hidelinks]{hyperref}
\usepackage[T1]{fontenc}
\usepackage[utf8]{inputenc}
\usepackage[usenames, dvipsnames, svgnames, table]{xcolor}
\usepackage[nottoc]{tocbibind}

\author{Io Boye Pinnerup}
\date{}

\newcommand{\f}[1]{\texttt{#1}}
\newcommand{\s}[1]{\textsc{#1}}
\newcommand{\ra}{\rightarrow}
\newcommand{\Ra}{\Rightarrow}
\newcommand{\LR}{\Leftrightarrow}
\newcommand{\txt}[1]{\textnormal{#1}}
\renewcommand{\.}{\cdot}
\renewcommand{\ss}[1]{\subsection{#1}}
\newcommand{\sss}[1]{\subsubsection{#1}}
\newcommand{\question}[1]{{\small \textcolor{Grey}{#1}}}

\pagestyle{fancy}
\fancyhead{}
\fancyfoot{}
\fancyhead[l]{I. B. Pinnerup}
\fancyhead[r]{\today}
\fancyfoot[c]{Page \thepage\ of \pageref{LastBody}}
\setlength{\headheight}{15pt}
\setlength{\headsep}{15pt}

\lstset{
    basicstyle=\footnotesize\ttfamily,
    keywordstyle = \color{Green},
    commentstyle = \color{Grey},
    identifierstyle = \color{Blue},
    stringstyle = \color{Purple},
    numbers = left,
    numberstyle = \tiny,
    stepnumber=1,
    showstringspaces = false,
    frame = single,
    captionpos = b,
    rulecolor = \color{Black},
    breaklines=true,
}

\newcommand{\code}[2]{\lstinputlisting[caption=\texttt{#2}, label=#1]{#1}}
\newcommand{\codepart}[4]{\lstinputlisting[caption=#2, firstline=#3,
                          lastline=#4, firstnumber=#3, label=#1#3#4]{#1}}
\newcommand{\coutput}[2]{\lstinputlisting[caption=\texttt{#2}, label=#1,
                        numbers=none, deletekeywords={*},
                        identifierstyle=\color{Black},
                        keywordstyle=\color{Black}]{#1}}
\newcommand{\fig}[3][]{
    \begin{figure}[H]
    \centering
    \includegraphics[width=\textwidth]{#2}
    \caption{#1}
    \label{#3}
    \end{figure}
}


\fancyhead[c]{Agenda}
\titlelabel{}

\begin{document}

\section{Agenda}

\ss{April 29th}

\section{Minutes}

\ss{April 15th}

\begin{enumerate}
  \item Status
    \begin{itemize}
      \item Grammar for propositions is implemented as mutually recursive 
      datatypes in SML

      \item Conversion to \f{string} is implemented for propositions
      
      \item Conversion to symbol syntax (\emph{not} \s{BoxProver}) is
      implemented for formulas

      \item Conversion of formulas to propositions is implemented
    \end{itemize}

  \item Discussion points
    \begin{description}
      {\color{DarkGrey}
      \item[Question:] Should quotation marks alternate between `p' and 
      ``p''?

      \item[Question:] Should disjunctions be negated with 'neither/nor'
      constructions?

      \item[Question:] Are non-bracketed implications fine as elements in
      lists, or should they be bracketed?

      \item[Question:] Are non-bracketed implications fine as final elements
      in \s{and also} constuctions? See result of tests for reference.

      \item[Question:] When an \s{and} list starts with an \s{and}-pair, 
      should this be split into two elements (which is what happens now), 
      or should the pair be kept as the first element?

      \item[Question:] Do bracketed statements and negations work in
      \s{pair}-constructions, or should \f{SINGLE} be exchanged for 
      \f{UNIT}?

      \item[Question:] Are we satisfied with how the propositions read in
      general? See result of tests for reference.}

    \end{description}

  \item Action items
    \begin{enumerate}[\HollowBox]
      \item[\CrossedBox] Update natural language proposition grammar 
      to account for the fact that
      implications only take \f{UNIT} and \f{NEGATION}, not \f{STAT}
      \item[\CrossedBox] Update natural language proposition grammar to 
      account for decisions
      \item \textcolor{Red}{Extend} \textcolor{Green}{Write} natural
      language grammar to proofs
        \begin{enumerate}[\HollowBox]
          \item Account for rules
          \item Account for references
          \item Account for preamble
        \end{enumerate}
    \end{enumerate}

  \item Misc
    \begin{description}
      \item[Decision:] At least for now, render formulae in symbol notation
      and focus on doing the proof steps in natural language.

      \item[Decision:] Create an "inclusive" rather than restrictive syntax 
      for proofs: it is important for the program to accept incorrect 
      proofs in order to correct the syntax. In other words: if students are
      to learn, they need to make mistakes.
    \end{description}
\end{enumerate}

\label{LastBody}
\end{document}
