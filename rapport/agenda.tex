\documentclass[a4paper]{article}

\usepackage{titlesec, fancyhdr, graphicx, float, nameref, framed, listings}
\usepackage{amsmath, enumerate, marvosym, fullpage, verbatim, subfiles}
\usepackage[hidelinks]{hyperref}
\usepackage[T1]{fontenc}
\usepackage[utf8]{inputenc}
\usepackage[usenames, dvipsnames, svgnames, table]{xcolor}
\usepackage[nottoc]{tocbibind}

\author{Io Boye Pinnerup}
\date{}

\newcommand{\f}[1]{\texttt{#1}}
\newcommand{\s}[1]{\textsc{#1}}
\newcommand{\ra}{\rightarrow}
\newcommand{\Ra}{\Rightarrow}
\newcommand{\LR}{\Leftrightarrow}
\newcommand{\txt}[1]{\textnormal{#1}}
\renewcommand{\.}{\cdot}
\renewcommand{\ss}[1]{\subsection{#1}}
\newcommand{\sss}[1]{\subsubsection{#1}}
\newcommand{\question}[1]{{\small \textcolor{Grey}{#1}}}

\pagestyle{fancy}
\fancyhead{}
\fancyfoot{}
\fancyhead[l]{I. B. Pinnerup}
\fancyhead[r]{\today}
\fancyfoot[c]{Page \thepage\ of \pageref{LastBody}}
\setlength{\headheight}{15pt}
\setlength{\headsep}{15pt}

\lstset{
    basicstyle=\footnotesize\ttfamily,
    keywordstyle = \color{Green},
    commentstyle = \color{Grey},
    identifierstyle = \color{Blue},
    stringstyle = \color{Purple},
    numbers = left,
    numberstyle = \tiny,
    stepnumber=1,
    showstringspaces = false,
    frame = single,
    captionpos = b,
    rulecolor = \color{Black},
    breaklines=true,
}

\newcommand{\code}[2]{\lstinputlisting[caption=\texttt{#2}, label=#1]{#1}}
\newcommand{\codepart}[4]{\lstinputlisting[caption=#2, firstline=#3,
                          lastline=#4, firstnumber=#3, label=#1#3#4]{#1}}
\newcommand{\coutput}[2]{\lstinputlisting[caption=\texttt{#2}, label=#1,
                        numbers=none, deletekeywords={*},
                        identifierstyle=\color{Black},
                        keywordstyle=\color{Black}]{#1}}
\newcommand{\fig}[3][]{
    \begin{figure}[H]
    \centering
    \includegraphics[width=\textwidth]{#2}
    \caption{#1}
    \label{#3}
    \end{figure}
}


\fancyhead[c]{Agenda}
\titlelabel{}

\begin{document}

\section{Agenda}

%\ss{DATE}
%\begin{enumerate}
%  \item Status
%    \begin{enumerate}[\HollowBox]
%    \end{enumerate}
%  \item Discussion points
%    \begin{description}
%    \end{description}
%  \item Action items
%  \item Misc
%\end{enumerate}

\ss{May 6th}
\begin{enumerate}
  \item Status
    \begin{enumerate}[\HollowBox]
      \item Add subsection on discharging to structural outline
      \item Finish validation of proofs
        \begin{enumerate}[\HollowBox]
          \item Handle rule validation, including checking premises
          \item Handle step validation
          \item[\Checkedbox] Handle id validation
          \item Handle proof validation
            \begin{enumerate}[\HollowBox]
              \item[\Checkedbox] Check all assumptions are discharged
              \item[\Checkedbox] Check that the conclusion in the last step
              is the goal
            \end{enumerate}
        \end{enumerate}
      \item Handle error feedback
        \begin{enumerate}[\HollowBox]
          \item Number of references inconsistent with rule requirements
          \item Pattern of references inconsistent with rule requirements
          \item A reference is to a proofstep in a closed box
          \item[\Checkedbox] The id given to a proofstep has already been 
          used
          \item[\Checkedbox] Not all assumptions are discharged
          \item[\Checkedbox] The conlclusion of the last step is not the goal
        \end{enumerate}
      \item Try printing with indentation
    \end{enumerate}
  \item Discussion points
    \begin{description}
      \item[Question:] What form should the regression testing take?
    \end{description}
  \item Action items
  \item Misc
\end{enumerate}

\newpage
\section{Minutes}

\ss{April 29th}
\begin{enumerate}
  \item Status
    \begin{enumerate}[\HollowBox]
      \item[\Checkedbox] Write natural language grammar for proofs
        \begin{enumerate}[\HollowBox]
          \item[\Checkedbox] Account for rules
          \item[\Checkedbox] Account for references
          \item[\Checkedbox] Account for preamble
        \end{enumerate}
    \end{enumerate}
  \item Discussion points
    \begin{description}
      \item[Question:] The implementation of proofs allows for errors, but
      presumably the formal grammar for proofs should not?
      \item[Answer:] The grammar on which the parser is built, should match
      the implementation and allow for errrors; this grammar should be seen
      as the an extention of a more restrictive one, that only allows for
      correctly structured proofs.
      \item[NB!] Some parts of validation are not context-free, e.g. that
      a specific rule can only be used on references which refer to formulas
      in specific relation to each other; thus a context-free grammar cannot
      be restrictive enough to only accept valid proofs.
      \item[Question:] What ouput should the validation function produce?\\
      My thought was feedback on each step (either in the terminal or an 
      output file), and if the proof is valid and closed, a file with the
      proof in \s{BoxProver} syntax, ready to be copy/pasted. What are your
      thought?
      \item[Answer:] Output should be file of either feedback or a valid
      proof in \s{BoxProver} syntax.
      \item[Question:] Is it considered an error if a proof goes on after
      the sought conclusion has been reached outside any boxes?
      \item[Answer:] Yes, this should be added to error handling.
      \item[SML question:] Since we want to give feedback on errors, should
      errors like using the wrong rule be implemented by raising and 
      handling exceptions, or simply incorporated in the regular code, so
      to speak?
      \item[Answer:] Giving feedback is correct use of the program, so
      exceptions should not be used for these types of errors.
      \item[SML question:] Is it possible to prompt the user for input by
      giving them an editable default string? I have not been able to figure
      out a way to do this, but it would be nice if the system could be 
      interactive without the user having to retype the whole proof (or
      alternatively ``just'' everything since the first error, every time
      they want to check their progress.
      \item[Answer:] There's a library for that.
      \item[Regression testing format] Forgotten, moved to next session.
    \end{description}
  \item Action items
    \begin{enumerate}[\HollowBox]
      \item Add subsection on discharging to structural outline
      \item Finish validation of proofs
        \begin{enumerate}[\HollowBox]
          \item Handle rule validation, including checking premises
          \item Handle step validation
          \item Handle id validation
          \item Handle proof validation
            \begin{enumerate}[\HollowBox]
              \item Check all assumptions are discharged
              \item Check that the conclusion in the last step is the goal
            \end{enumerate}
        \end{enumerate}
      \item Handle error feedback
        \begin{enumerate}[\HollowBox]
          \item Number of references inconsistent with rule requirements
          \item Pattern of references inconsistent with rule requirements
          \item A reference is to a proofstep in a closed box
          \item The id given to a proofstep has already been used
          \item The conlclusion of the last step is not the goal
        \end{enumerate}
      \item Try printing with indentation
    \end{enumerate}
  \item Misc.
    \begin{description}
      \item[Question:] Fritz, you seemed to want to change how I deliver the
      agenda?
      Should I edit the calendar event instead, or keep it on gitHub?
      \item[Answer:] Material is to be sent wednesday night or thursdag 
      morning.
      \item[NB!] Boxes in fitch-style proofs discharge assumptions when
      closing a box, and thus independently of the conclusion drawn from it,
      whereas existing nomenclature uses ``discharge'' when applying a rule
      to a box and thus drawing conclusions based on findings under the
      assumption in question.
    \end{description}
\end{enumerate}

\ss{April 15th}
\begin{enumerate}
  \item Status
    \begin{itemize}
      \item Grammar for propositions is implemented as mutually recursive 
      datatypes in SML

      \item Conversion to \f{string} is implemented for propositions
      
      \item Conversion to symbol syntax (\emph{not} \s{BoxProver}) is
      implemented for formulas

      \item Conversion of formulas to propositions is implemented
    \end{itemize}

  \item Discussion points
    \begin{description}
      {\color{DarkGrey}
      \item[Question:] Should quotation marks alternate between `p' and 
      ``p''?

      \item[Question:] Should disjunctions be negated with 'neither/nor'
      constructions?

      \item[Question:] Are non-bracketed implications fine as elements in
      lists, or should they be bracketed?

      \item[Question:] Are non-bracketed implications fine as final elements
      in \s{and also} constuctions? See result of tests for reference.

      \item[Question:] When an \s{and} list starts with an \s{and}-pair, 
      should this be split into two elements (which is what happens now), 
      or should the pair be kept as the first element?

      \item[Question:] Do bracketed statements and negations work in
      \s{pair}-constructions, or should \f{SINGLE} be exchanged for 
      \f{UNIT}?

      \item[Question:] Are we satisfied with how the propositions read in
      general? See result of tests for reference.}

    \end{description}

  \item Action items
    \begin{enumerate}[\HollowBox]
      \item[\CrossedBox] Update natural language proposition grammar 
      to account for the fact that
      implications only take \f{UNIT} and \f{NEGATION}, not \f{STAT}
      \item[\CrossedBox] Update natural language proposition grammar to 
      account for decisions
      \item \textcolor{Red}{Extend} \textcolor{Green}{Write} natural
      language grammar to proofs
        \begin{enumerate}[\HollowBox]
          \item Account for rules
          \item Account for references
          \item Account for preamble
        \end{enumerate}
    \end{enumerate}

  \item Misc
    \begin{description}
      \item[Decision:] At least for now, render formulae in symbol notation
      and focus on doing the proof steps in natural language.

      \item[Decision:] Create an "inclusive" rather than restrictive syntax 
      for proofs: it is important for the program to accept incorrect 
      proofs in order to correct the syntax. In other words: if students are
      to learn, they need to make mistakes.
    \end{description}
\end{enumerate}

\label{LastBody}
\end{document}
