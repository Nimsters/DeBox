\documentclass[manual.tex]{subfiles}
\begin{document}
Because of the desire to give feedback, the surface syntax for proofs in
natural language was designed to be inclusive and allow for errors rather
than restrict acceptable proofs to correct ones.

\sss{Inclusive grammar for proofs}
To ensure acceptance of both correct and errouneous proofs, we 
 \chk{define} the grammar for proofs in natural language based on the
 abstract syntax for proofs rather than the formal surface syntax.
 The definition is shown in \fig{nlproof}
 and will accept almost any 
 \emph{rule;conclusion;reference pattern}-combination representable
 by the abstract \f{proofstep}. For instance, it accepts the use
 of \emph{law of the excluded middle} with any number of references, or
 the use of \emph{disjunction-introduction} to conclude an implication.
 One exception to the grammars match with the abstract syntax is
 the distinction between
 discharges and all other steps; \f{DISCHARGE} is the only non-terminal
 that allows the acceptance of a step without a formula, and \f{DIS} is
 \emph{not} included in the set of rules that can be used in acceptable
 steps \emph{with} a formula. Furthermore, \f{DISCHARGE} steps are the only
 steps -- apart from the very last one -- that are accepted without a step
 ID, and they are \emph{only} accepted without one. 
 
 Structurally, all non-discharge steps are composed of an \f{ARGUMENT} and
 a \f{CONCLUSION}, and both allow variation in composition.
 Furthermore, a distinct non-terminal is defined for every 
 \f{rule} in the abstract syntax, which allows for later addition of
 variant rule representations (e.g. shorthand), without the need to
 redefine the rest of the syntax definition.

 In terms of English grammar, the proof syntax accepts lists both with and
 without Oxford comma\footnote{Also known as the serial comma; the term
 denotes the comma before the \chk{connective} in a list. 
 With Oxford comma: `propositions, formulae, and proofs'. 
 Without: `propositions, formulae and proofs'.},
 but ensures the use of obligatory comma after an \chk{adverbial phrase}
 when it precedes the main clause. 

\begin{figure}[!ht]
\scriptsize
\subfile{grammar_inclusive}
\caption{An inclusive grammar for proofs by natural deduction written in
natural language}
\label{nlproof}
\end{figure}

\FIX{\sss{Design decisions: Implicit boxes, explicit discharges}
 Something something}


\sss{Inclusive grammar for formulae}
In contrast to the general proof syntax, we base our grammar for
 formulae on the formal syntax. In fact, but for one change \ours\ would 
 classify as an extention of it.
 As shown in \fig{nlform}, \abs\ is no longer signified by
 \f{bot}, as is the case in the formal syntax, but rather by both a sequence 
 of \f{ASCII} characters (\verb+_|_+) and a Unicode character (\(\bot\)). 
 The same goes for the operators, except for implication which has an
 additional regular character signifier, to allow for individual preferences
 in typing implication arrows. Having multiple signifiers for each
 operator, and \chk{the \abs\ formula}, ensures that the grammar
 accepts both formulae with \s{ASCII} and with Unicode operators.
 Furthermore, having a distinct non-terminal, and thus \f{token}
 for each of the aforementioned syntactic elements makes it
 very simple to extend the grammar with more signifiers later on, or change
 the existing ones, without having to rewrite any of the \f{PROP} 
 non-terminals.

\begin{figure}[!hb]
\subfile{grammar_symbols}
\caption{A grammar for propositional formulae using symbol notation}
\label{nlform}
\end{figure}

\sss{Design decisions: logical operators and linguistic ambiguity}
When formulating a grammar for logical formulae in natural language, it
 might seem like an obvious choice to use the english words
 for the logical terms
 (\emph{and}, \emph{or}, \emph{not}, \emph{if ... then},  and \emph{\abs})
 as signifiers instead of the \chk{symbol} representation used. However, 
 the English language, \chk{like most - if not all - natural languages}, is
 ambiguous. While a statement like:
 \begin{quote}
    The dog barks \emph{and} the cat meows.
 \end{quote}
 seems unambiguous, the same is not the case for the following:
 \begin{quote}
    The dog barks \emph{or} the cat meows.
 \end{quote}
 Is the \emph{or} to be interpreted as inclusive 
 (i.e. `either ... or ... but not both') or exclusive
 (i.e. `at least one of ... and ..., possibly both')? 
 The intuition for most will be exclusive, but the logical disjunction
 operator is inclusive. And what about more complex composite formulae,
 and the negation of those? 
 
 As it turns out, not even the simple conjunction can be considered 
 unambiguous; in a study\cite{negation} of the understanding and
 formulation of compound assertions (conjunctions, disjunctions, 
 conditionals, and negations of these) in English, only 86\% of the
 participants interpreted the conjunctions in accordance with the
 standard semantics of propositional logic\cite[p.~4]{negation}.
 The study also
 showed that the interpretation of disjunctions as either
 inclusive or exclusive varied, with only 9\% of participants reading
 them as inclusive, against 59\% exclusive\cite[\emph{ibid.}]{negation}.
 Furthermore, this bias towards an exclusive interpretation was completely
 reversed in the case of negated disjunctions where 86\% gave an inclusive
 interpretation:
 \begin{quote}
 The compound, \emph{A or B},
 was treated as exclusive [...] rather than inclusive [...]
 in affirmations, but as inclusive [...] rather than exclusive [...] 
 in negations. The compound, \emph{If A then B}, was
 often treated as a biconditional [...] in affirmations, but
 never in negations.\cite[p.~5]{negation}
 \end{quote}
 Since interpretation not only varies between individuals but also 
 depends on context, simply defining e.g. disjunctions as one or the other
 will result in enforcing a counterintuitive interpretation in many cases,
 regardless of which interpretation is picked.
 
 On top of that, standard English, while allowing for
 complex sentences using subordinate clauses, does not lend itself well to
 compound contructions nested to arbitrary depth. Although there is no 
 inherent precedence distinction between conjunctions and disjunctions, it
 \emph{is} possible to guide the correct reading, and thus base a fixed
 nesting structure on this, by using 
 subordinators of different length:
 \begin{quote}
 \ttfamily
    p \emph{and} q \emph{or else} r \hfill (p \(\land\) q) \(\lor\) r~ \\
    p \emph{and also} q \emph{or} r\hfill p \(\land\) (q \(\lor\) r)
 \end{quote}
 This solution, however, does not work for arbitrarily deep nestings, nor
 does it solve the inclusive/exclusive problem for disjunctions.

 Rather than introduce a complex new set of rules to narrow (and possibly
 counter) the intuitive understanding of conjunctions, disjunctions,
 conditionals, and the negations of these, the syntax uses
 \chk{symbolic} notation, for which there already exists a set standard
 for unambiguous interpretation and which is an inherent part of
 propositional logic and thus necessarily on the syllabus of the course.


\end{document}
