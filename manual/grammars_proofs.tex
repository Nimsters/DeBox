\documentclass[manual.tex]{subfiles}
\begin{document}
\ss{Writing proofs}

All proofs follow the same structure, consisting of three main parts:
\begin{enumerate}
\item The title followed by a colon.
\item The sequent to be proven.
\item The steps by which the conclusion is reached.
\end{enumerate}

\paragraph{The title} can be any string consisting of alphanumerical 
characters, hyphens, and underscores, and \emph{must} be followed by
a colon. Like this:
\begin{figure}[!hb]
example\_title:
\caption{Example of a proof title.}
\label{ex:title}
\end{figure}

\paragraph{The sequent} should be phrased in one of three different ways, 
 depending on the number of premises:

\begin{itemize}
\item {\bf None:}\\ We wish to prove \f{<formula>}.
\item {\bf One:}\\ From the premise \f{<premise>}, we wish to prove 
    \f{<formula>}.
\item {\bf Two or more:}\\ From the premises \f{<premise 1, ...,>} and 
    \f{<premise n>}, we wish to prove \f{<formula>}.
\end{itemize}
The sequent (like each step) \emph{must} end with a period. Formulae (both
premises and intended goal) are written using strings (alphanumeric 
characters, hyphens, underscores) for propositional atoms, and using
symbols for logical operators and to signify absurdity. 
These can be written using
 \f{ASCII} characters or the appropiate Unicode character -- see figure
 \ref{operators} below. 
 For implication, both \f{=>} and \f{->} are accepted,
 to allow for individual preferences. 
 
\begin{figure}[!hb]
\subfile{grammar_symbols}
\caption{Logical operators and $\bot$}
\label{operators}
\end{figure}

\sss{Proof steps}
Each step is the natural language equivalent of a line in a 
 fitch-style proof\footnote{With the exception of assumption discharges,
 since the domain of an assumption is illustrated by a framing box in
 fitch-style notation}, and is either structural or deductive.

\paragraph{Structural steps} provide context for following steps, and come
in three varieties:
\begin{itemize}
  \item Stating premises\\
      $\.$ All premises must be stated before any other steps are taken.\\
      $\.$ Premises must be stated in the order in which they appear in
      the sequent.
  \item Stating assumptions\\
      $\.$ All assumptions must be discharged before the proof is closed.
  \item Discharging assumptions\\
      $\.$ Only the most recently made \emph{undischarged} assumption may 
      be disharged.
\end{itemize}

\paragraph{Deductive steps} advance the proof by applying one of the 
rules of natural deduction in the current context. Each rule has exactly
 one equivalent wording in English, to emphasize the formalism of natural
 deduction proofs, and these can be found in the Quick-reference on 
 page~\pageref{qr}.

Rather than refer to previous steps by number, each step is given a unique
 reference id, listed in square brackets after the formula obtained in
 the step. These ids follow the same rules as titles and atoms, and are
 required in all steps except discharges (which optain no new formulae)
 and the concluding step of a proof.

Some rules reference a range of lines (\f{[<first>]-[<last>]}). These are
the equivalent of boxes in fitch-style notation, and \f{[<first>]} should
always refer to an assumption while \f{[<last>]} should always refer to
the line immidiately before the discharge of that assumption.

To illustrate all this, figure \ref{ex:proof} shows an example of a proof:

\begin{figure}[!hb]
\input{valid.nl}
\caption{Example of a proof title.}
\label{ex:proof}
\end{figure}

{\bf Note:}\\
 In terms of English grammar, the proof syntax accepts lists both with and
 without Oxford comma\footnote{Also known as the serial comma; the term
 denotes the comma before the connective in a list. 
 With Oxford comma: `propositions, formulae, and proofs'. 
 Without: `propositions, formulae and proofs'.},
 but ensures the use of obligatory comma after an adverbial phrase
 when it precedes the main clause (e.g. in `By applying ... {\bf ,} we get
 ...'). 

\clearpage
\section{Parse-error checklist}
\begin{itemize}
\item Parse error in col 0:
    \begin{enumerate}[{\bf ?}]
      \item Did you forget to end the previous line with a period?
      \item Did you mis-spell the rule on this line?
    \end{enumerate}
\item Parse error on last line, col after formula:
    \begin{enumerate}[{\bf ?}]
      \item Did you use `, we get' but have no \f{[<ref>]} for the line?
      \item Did you use `, we conclude' and have a \f{[<ref>]} for the line?
    \end{enumerate}
\item General:
    \begin{enumerate}[{\bf ?}]
      \item Do you have double spaces somewhere on the line?
    \end{enumerate}
\end{itemize}

\clearpage
\section{Quick-reference}\label{qr}

\paragraph{Title, atoms, and references} ~\\
Alphanumeric characters, hyphens, underscores.
Title must be followed by a `:'.

\paragraph{Sequents} ~
\begin{itemize}
\item We wish to prove \f{<formula>}.
\item From the premise \f{<premise>}, we wish to prove \f{<formula>}.
\item From the premises \f{<premise 1, ...,>} and 
        \f{<premise n>}, we wish to prove \f{<formula>}.
\end{itemize}

\paragraph{Symbols} ~

\begin{figure}[H]
\subfile{grammar_symbols}
\end{figure}

\paragraph{Steps} ~

\begin{figure}[H]
\scriptsize
\subfile{grammar_restrictive}
\caption{Proof steps}
\label{steps}
\end{figure}



\end{document}
